% Author: PokMan Ho pok.ho19@imperial.ac.uk
% Script: admin.tex
% Desc: admin section
% Input: none
% Output: none
% Arguments: 0
% Date: Jan 2020

\documentclass[../note.tex]{subfiles} %% use packages & commands as this main file

\begin{document}

\section{Administrative issues}

\subsection{R environment}
It is crucial to clean the environment just to make sure none of the previous operations have any chance to mess up your current code, making your analyses non-reproducible.
\begin{code}
rm(list=ls(all=T))\\\#rm all\\\\
rm(list=ls(pattern="temp"))\\\#rm data names contain pattern "temp"
\end{code}

\subsection{Sample data statistical test/plots parts}
The author chose to use arbitrary data for demonstration purposes.  Codes for generating such data sets were showed and explained below:

One numeric dependent + two categorical unordered independent + one numeric independent
\begin{code}
set.seed(998)\\\# fix random number generation algorithm, for code reproducibility\\\\

a<-as.data.frame(matrix(nrow = 1e3, ncol = 4))\\\# initialize empty dataframe\\\\

colnames(a)=c("dep", "indep\_1", "indep\_2", "indep\_3")\\\\

a\$dep<-runif(nrow(a))\\\# sample number of values meet number of rows of dataframe \textbf{a} from uniform distribution between 0 and 1\\\\

a\$indep\_1<-c("a","b")\\\# repeating "a", "b" pattern until fill up the dataframe (length difference must be in multiples)\\\\

a\$indep\_2<-sample(c("a","c","d","e"),nrow(a), replace = T)\\\# sample number of values meet number of rows of dataframe \textbf{a} from designated pool with element replacement after each element selection\\\\

a\$indep\_3<-rnorm(nrow(a))\\\# sample number of values meet number of rows of dataframe \textbf{a} from normal distribution with mean 0, sd 1
\end{code}

\subsection{Colours}
Colours used in this note were colour-blinded friendly, collected from various sources in the internet:
\begin{code}
cbp <- c("\#000000", "\#E69F00", "\#56B4E9", "\#009E73", "\#0072B2", "\#D55E00", "\#CC79A7", "\#e79f00", "\#9ad0f3", "\#F0E442", "\#999999", "\#cccccc", "\#6633ff", "\#00FFCC", "\#0066cc")
\end{code}

\end{document}